% !TEX root = ../section1.tex
\section{Aldehydes and Ketones}

E.g.
\img{4-1}

\subsection{Reactivity}

\img{4-2}

There is decreasing reactivity from formaldehyde to ketones as the the indictive
effect lowers the \ce{\delta+} value on the central carbon atom, and the steric
bulk around the carbon atom increases.

\subsection{Preparation of Aldehydes and Ketones}

\begin{enumerate}[label=\alph*)]

  \item Alkynes with acid
    \img[A terminal alkynes always produces a methyl ketone]{4-3}

  \item Oxidation of alcohols (\ce{CrO3, PCC})

    \begin{enumerate}[label=\roman*)]

      \item 1\de\ Alcohols
        \img{4-4}

      \item 2\de\ Alcohols
        \img{4-5}

    \end{enumerate}

    Oxidation of an alcohol followed by grignard reagent is a powerful synthetic
    method.

  \item Ozonolysis of alkenes (see section 2 for mechanism)

    \begin{enumerate}[label=\roman*)]

      \item\
        \img{4-6}

      \item\
        \img{4-7}

      \item\
        \img{4-8}

    \end{enumerate}

  \item Friedal Crafts Acylation

    \img{4-9-2}

    This begins with the formation of the acylium ion

    \img{4-9-1}

    Followed by the mechanism

    \img{4-9-3}

    R\super{2} = H or an e\super{-} donating group such as Me, OMe or a halogen.

    R\super{1} = anything except H.

\end{enumerate}

\subsection{Reactions of Carbonyl Compounds}

The general pattern of reactivity is:

\img{4-10}

\begin{enumerate}[label=\alph*)]

  \item Reduction of carbonyl compounds

    \begin{enumerate}[label=\roman*)]

      \item Using \ce{H2, Pd-C}

        Reduces aldehydes and ketones to alcohols, esters are only slowly
        reduced to alcohols. Inexpensive and no by products. Alkyne and alkene
        unsaturation is readily reducted to alkane.

        \img{4-11}

      \item Using metal hydrides

        Requires coordination to O or N and therefore does not reduce alkyne
        or alkene unsaturation. I.e. Chemoselective

        \img{4-13}

      \item Sodium borohydride

        Reduces aldehydes and ketones by not esters or amides.

        \img{4-12}

        Mechanism

        \img{4-14}

    \end{enumerate}

  \item Addition of C--nucleophiles to give

    \begin{enumerate}[label=\roman*)]

      \item Cycanohydrins from HCN/KCN

        \img{4-15}

        Use in synthesis is:

        \img{4-16}

      \item Alcohols from a grignard reagent.
        \img{4-17}

      \item Alkene from a Wittig reagent

        \img{4-18}

        The ylide is generated as so:

        \img{4-19}

        The overall mechanism is:

        \img{4-20}

    \end{enumerate}


  \item Addition of O-nucleophiles

    \begin{enumerate}[label=\roman*)]

      \item Acetals from an alcohol with pTsOH

        \img{4-21}

        Mechanism:

        \img{4-22}

        Acetals can be used to protect \ce{C=O} of an aldehyde or ketone.

      \item Preparation of Esters with mCPBA; The Baeyer-Villiger Oxidation

        \img[Ketones only]{4-23}

        Mechanism

        \img{4-24}

        Another example is:

        \img{4-25}

    \end{enumerate}

  \item Addition of N-nucleophiles

    \img{4-26}

    This is the same mechanism as the formation of acetals except with the
    elimination of water at the end.

  \item C-nucleophiles from aldehydes and ketones.

    \begin{enumerate}[label=\roman*)]

      \item $\beta$-halocarbonyl compounds with \ce{Br2}

        This proceeds via the acid catalysed formation of an enol.

        \img{4-27}

        E.g. $\alpha$-bromination of carbonyl compounds

        \img{4-28}

        Here there is a low conversion of carbonyl compound into enol by
        acid catalysis However the reaction proceeds as the formation of
        the strong \ce{C-Br} bond is irreversible.

      \item Carboxylic acid with NaOH and \ce{I2} (haloform reaction)

        \img{4-30}

        Mechanism:

        \img{4-29}

      \item Aldol addition

        \img{4-31}

        Aldol condensation is an aldol addition followed by elimination of water.

        \img{4-32}

        When a ring is formed the aldol addition product rapidly eliminates
        to give the enone:

        \img{4-33}

    \end{enumerate}

\end{enumerate}