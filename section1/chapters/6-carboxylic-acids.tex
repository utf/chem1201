% !TEX root = ../section1.tex
\section{Carboxylic Acids}

\subsection{General}

E.g.
\img{6-1}

Reactivity of carboxy derivatives towards nucleophiles:
\img{6-2}

They decrease in reactivity as:
\img{6-3}

The inductive effect of Cl makes acid chlorides powerful electrophiles
The mesomeric effect of O and N lower the electrophilicty of the carbonyl group.

\subsection{Preparation of carboxylic acids}

\begin{enumerate}[label=\alph*)]

  \item Oxidation of an alcohol or an aldehyde (Jones' Reaction)
    \img{6-4}

  \item Hydrolysis of nitrile group (NaOH then dil \ce{H2SO4})
    \img{6-5}

  \item Addition of \ce{CO2} to a grignard reagent
    \img{6-6}

\end{enumerate}


\subsection{Reactions of carboxylic acids}

\begin{enumerate}[label=\alph*)]

  \item Ester from an alkyl iodide
    \img{6-7}

  \item Ester from an alcohol with catalytic \ce{H2SO4} (Fischer Esterification)

    \img{6-8}

  \item Lactone by cyclisation of a hydroxy acid or ester: 5 and 6 membered rings
    are easily formed under acid catalyst conditions and where both are possible,
    the 5 membered ring is favoured.

    E.g. 1
    \img[IR vmax = 1760]{6-9}

    E.g. 2
    \img[IR vmax = 1740]{6-10}

  \item Acid chloride with \ce{SOCl2}
    \img{6-11}

\end{enumerate}