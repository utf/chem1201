% !TEX root = ../main.tex
\section{Alcohols}

\subsection{General}

\img{2-1}

Owing to conjugation of \ce{O} via the sp\super{2} carbon, phenols and enols
behave differently and neither is referred to as an alcohol.

\img{2-2}

\ce{CH3CH2OH} cannot be oxidized as there is no $\alpha$--H

\subsubsection{Physical Properties}

The electro-negativity of \ce{O} means that alcohols are feebly acidic unlike
amines that are only ever feebly basic. Alcohols are also feebly basic (\ce{O}
is less nucleophilic than \ce{N}). They are also extensively hydrogen bonded
which gives them much higher boiling points than alkyl halides.

\subsection{Preparation of Alcohols}

\begin{enumerate}

  \item Reduction of \ce{C=C} compounds.\\
    \img{2-3}
    Examples:
    \img{2-4}
    \img[Note the double bond is unaltered.]{2-5}

  \item Addition of grignard (\ce{RMgX}) to a carbonyl compound.
    \img{2-6}
    \textbf{Mechanism}
    \img{2-7}

    \begin{description}
      \item[Example 1.] Alcohols from aldehyde's.\\
        \img{2-8}

      \item[Example 2.] Alcohols from ketone's.\\
        \img{2-9}

      \item[Example 3.] Alcohols from esters.\\
        \img{2-10}
    \end{description}

    Esters only give alcohols with grignard reagents because the inductive
    effect increases reactivity but mesomeric effects are greater therefore
    the ketone's \ce{C=O} is more reactive than an ester \ce{C=O}.

    \img{2-11}

    Grignard reagents are destroyed by groups with an exchangeable H e.g. OH,
    SH, NHR, COOH and thus require protective groups, e.g. silicon for an
    alcohol.\\

    For example:

    \imginline{2-12} cannot go via the intermediate
    \imginline{2-13} as the OH would destroy the
    grignard reagent formed. Therefore instead, silicon is used as a protective group.

    \img{2-14}

  \item Hydroboration of alkenes (delivers \ce{OH} to the less substituted
    \ce{C}).
    \img{2-15}
    Mechanism
    \img{2-16}

  \item Oxymercuration of alkenes (delivering of OH to the more substituted C)
    \img{2-17}
    Mechanism
    \img{2-18}

\end{enumerate}

\subsection{Reactions of Alcohols}

\subsubsection{Reaction at the alcohol oxygen atom.}

\begin{enumerate}[label=\alph*)]

  \item Formation of the alkoxide (Na, NaH)

    With a strong base, the acidic H is lost and the alkoxide is formed.
    Grignard reagents must also be protected from this e.g.

    \begin{center}
      \ce{ROH + Na -> RO- Na+ + 1/2 H2}\\
      \ce{ROH + Na+ H- -> RO- Na+ + H2}
    \end{center}

    Alkoxides are good bases and good nucleophiles except tBuOH and
    3\super{$\circ$} alcohols, which are good bases but non-nucleophilic due
    to their steric hindrance. NaH acts only as a base and is not a reducing
    agent.

  \item O-Alkylation (alkoxide + alkyl halide)

    \img[Williamson ether synthesis]{2-19}

  \item O-Acylation (alcohol + acid chloride)

    \img{2-20}

  \item O-sylfonylation (p-TsCl + Pyridine)

    \img{2-21}
    Mechanism
    \img{2-22}

    Tosylate is a very good leaving group and can be displaced by many
    nucleophiles including all halides.
    \img{2-23}
    DMSO is \ce{Me2S=O}, a very popular solvent that gives fast rates of reaction.

\end{enumerate}

\subsubsection{Displacement at the alcohol carbon atom}

Activation of the OH group is the first step, in all cases a good leaving group
(HOX) is generated.

\begin{enumerate}[label=\alph*)]

  \item Conversion of ROH into RCl (alkyl chloride)

    \img{2-24}

    The first transition state formed contains an \ce{O-S} bond. This
    is followed by elimination of chlorine and loss of a proton. Then
    \ce{SN2} displacement of the activated carbon atom occurs.

  \item Conversion of ROH into RBr (alkyl bromide)

    \img{2-25}

    The limitations of using HCl/HBr to prepare alkyl halides are:

    \begin{itemize}

      \item 2\de\ and 1\de\ alcohols require forcing conditions (100
        -120 \de C)

      \item incompatibility of any unsaturated sites, which will react.

      \item Likely to undergo rearrangement

    \end{itemize}

  \item Rearrangment using HCl/HBr 2\de\ carbocation

    \img[Mechanism results in the formation of a 3\de\ carbocation]{2-26}

    1,2 Hydride shifts are common where the resulting carbocation is more
    stable than the initial one.

\end{enumerate}

\subsubsection{Eliminations of Alcohols: Formal loss of water}

\begin{enumerate}[label=\alph*)]

  \item Where a carbocation is not trapped by a nucleophile (and does not
    rearrange) an elimination can occur. tBuOH reacts with \ce{H2SO4} to give
    a 3\de\ carbocation, which then deprotonates to give 2-methylbutane.
    The conditions favour the most substituted alkene.

    E.g.

    \begin{enumerate}[label=\roman*)]

      \item\ \img{2-27}

      \item\ \img{2-28}

    \end{enumerate}

    Limitations are that 2\de\ and 1\de\ alcohols require heating that may
    promote side reactions including rearrangements.

    Alternatives are elimination using \ce{POCl3} and pyridine and conversion
    of the alcohol into the tosylate followed by elimination with tBuOK.


  \item E\sub{2} elimination using \ce{POCl3} and Pyridine (at O \dec)

    \img{2-29}

  \item E\sub{2} elimination of the tosylate using tBuOK

    This is especially useful when the compound is sensitive to acidic reagents
    including (\ce{POCl3})

    \img{2-30}

\end{enumerate}

\subsubsection{1,2 Elimination across the \ce{C-O}: Oxidation of Alcohols}

Oxidation can be loss of H, loss of e\super{-} or gain of O.

\begin{enumerate}[label=\alph*)]

  \item Chromium (VI) reagents

    \begin{enumerate}[label=\roman*)]

      \item Dilute dichromate with dilute \ce{H2SO4}

        \img{2-31}

        Over oxidation of 1\de\ alcohols to RCOOH occurs. Any Cr(VI) reagent
        is good for 2\de\ alcohols.

      \item Pyridinium Chlorochromate (PCC)

        Good for converting 1\de\ alcohols to aldehydes and 2\de\ to ketones,
        with little over oxidation.

        \img[Formation of PCC]{2-32}

      \item \ce{CrO3} in aqueous \ce{H2SO4}: Jones reagent.

        Oxidises 2\de\ alcohols to the ketone and 1\de\ to the acid. The
        mechanism of Cr(VI) oxidations all involve formation of a
        chromate ester that undergoes E\sub{2} elimination.

        \img{2-33}

    \end{enumerate}

  \item Cleavage of 1,2-diols by sodium periodate, \ce{NaIO4}
    \img[A central \ce{C-C} bond is broken as part of the oxidation process]{2-34}

\end{enumerate}