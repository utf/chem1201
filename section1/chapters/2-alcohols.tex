% !TEX root = ../main.tex
\section{Alcohols}

\subsection{General}

\img{2-1}

Owing to conjugation of \ce{O} via the sp\super{2} carbon, phenols and enols
behave differently and neither is referred to as an alcohol.

\img{2-2}

\ce{CH3CH2OH} cannot be oxidized as there is no $\alpha$--H

\subsubsection{Physical Properties}

The electro-negativity of \ce{O} means that alcohols are feebly acidic unlike
amines that are only ever feebly basic. Alcohols are also feebly basic (\ce{O}
is less nucleophilic than \ce{N}). They are also extensively hydrogen bonded
which gives them much higher boiling points than alkyl halides.

\subsection{Preparation of Alcohols}

\begin{enumerate}

	\item Reduction of \ce{C=C} compounds.\\
		\img{2-3}
		Examples:
		\img{2-4}
		\img[Note the double bond is unaltered.]{2-5}

	\item Addition of grignard (\ce{RMgX}) to a carbonyl compound.
		\img{2-6}
		\textbf{Mechanism}
		\img{2-7}

		\begin{description}
			\item[Example 1.] Alcohols from aldehyde's.\\
			\img{2-8}

			\item[Example 2.] Alcohols from ketone's.\\
			\img{2-9}

			\item[Example 3.] Alcohols from esters.\\
			\img{2-10}
		\end{description}

		Esters only give alcohols with grignard reagents because the inductive
		effect increases reactivity but mesomeric effects are greater therefore
		the ketone's \ce{C=O} is more reactive than an ester \ce{C=O}.

		\img{2-11}

		Grignard reagents are destroyed by groups with an exchangeable H e.g. OH,
		SH, NHR, COOH and thus require protective groups, e.g. silicon for an
		alcohol.\\

		For example:

		\imginline{2-12} cannot go via the intermediate
		\imginline{2-13} as the OH would destroy the
		grignard reagent formed. Therefore instead, silicon is used as a protective group.

		\img{2-14}

	\item Hydroboration of alkenes (delivers \ce{OH} to the less substituted
	\ce{C}).

	\img{2-15}

	\textbf{Mechanism}

	\img{2-16}

\end{enumerate}

\subsection{Reactions of Alcohols}
\subsubsection{Reaction at the alcohol oxygen atom.}