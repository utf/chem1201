% !TEX root = ../main.tex
\section{Ethers and Epoxides}

\subsection{Ethers}

Ethers are good solvents as they are chemically inert but slightly polar. Old
bottles become oxidised by the air to give explosive peroxides.

Examples of ethers are:

\img{3-1}

\subsubsection{Preparation of ethers}

\begin{enumerate}[label=\alph*)]

  \item For symmetrical ethers
    \img{3-2}

  \item Williamson ether synthesis; the most general route
    \img[X= Br, I and OTs as long as there is not too much steric hindrance]{3-3}
    \ce{R-O-} is a very powerful nucleophile.

\end{enumerate}

\subsubsection{Reactions of Ethers: Cleavage by HI}

\ce{I-} attacks the less substituted (less sterically hindered) \ce{\alpha-C}

\img{3-4}

\subsection{Epoxides}

Epoxides are strained and highly reactive 3-membered ring ethers.

\subsubsection{Synthesis of Epoxides}

\begin{enumerate}[label=\alph*)]

  \item Cyclisation of halohydrins: Intramolecular Williamson ether synthesis
    \img{3-5}
    There is ring strain as the angles are normally 109\de\ and in epoxides
    they are 60\de .

  \item Epoxidation of alkenes

    The configoration of the alkene is retained in the epoxide.
    \img[Does not work on many alkenes]{3-6}
    3 membered rings are favoured as 4 membered rings have a lower entropy
    factor and therefore there is less chance of ring closure.

\end{enumerate}

\subsubsection{Reactions of Epoxides}

\subsubsection{Patterns of reactivity}

\begin{enumerate}[label=\roman*)]

  \item  All reagents except acid
    \img{3-7}

  \item Acid attack by HX
    \img{3-8}

\end{enumerate}

\subsubsection{Nucleophilic attack of epoxides}

\begin{enumerate}[label=\roman*)]

  \item Attack by C--Nucleophiles

    Powerful in synthesis as a new \ce{C-C} bond is generated.

    \begin{description}

      \item[Example 1.] Cyanide
        \img{3-9}

      \item[Example 2.] Alkynyl anions
        \img{3-10}

      \item[Example 3.] Grignard reagents
        \img{3-11}

    \end{description}

  \item Attack by Hydroxide, \ce{O-, S-} and \ce{N-}  nucleophiles.

    These all react according to the first pattern of reactions.

\end{enumerate}