% !TEX root = ../section2.tex
\section{Alkyl Halides (Haloalkanes)}

The chemistry of halogens is dominated by the nature of the halogen bond. Due to
the electronegativity of the halogens, bond polarisation occurs through inductive
effects \ce{H3C^{$\delta$+}-X^{$\delta$-}}. This means that the carbon is
electropositive or electrophilic and therefore reactions involving heterolysis
of the \ce{C-X} bond will lead to \ce{X-}. THis group, X, is termed the leaving
group.

Within the halides the relative abilities to function as leaving groups are:
I > Br > Cl > F. This is because, although F is more electronegative than I (due
to greater shielding), orbital overlap between C and F is very effective and forms
a strong bond (teflon). The \ce{C-I} bond is much weaker.

\subsection{Notes for alcohols}

\ce{HO-} is not a great leaving group, however it can be encouraged to leave in two
ways:

\begin{enumerate}[label=\alph*)]

  \item Protonation with an acid source (\ce{H2SO4, HCl}, etc).
    \img{4-1}

  \item Conversion to a better leaving group
    \img[\ce{TsO-} is a good leaving group (better than \ce{I-})]{4-2}

\end{enumerate}