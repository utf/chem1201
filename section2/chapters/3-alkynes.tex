% !TEX root = ../section2.tex
\section{Alkynes}


Alkynes have a single carbon-carbon triple bond and have the general formula
C\sub{n}H\sub{2n-2}. Ethyne has a linear arrangement of the four atoms. The hybrid
atomic orbitals in each carbon are constructed using the 2s orbital and only one
of the 2p orbitals to give two sp hybrid atomic orbitals leaving the 2py and 2pz
orbitals. The two hybrid sp orbitals are construced at 180\de\ to each other.
Each remaining p orbital contains a single electron, sideways overlap of each of
these on the neighbouring carbon atoms forms two $\pi$ bonds.

\img{3-1}

\subsection{Reactions of Alkynes}

\begin{enumerate}[label=\alph*)]

  \item Bromination

    \img{3-2}

    The reaction can be controlled to stop to give \textbf{(A)}. Xs reagent is needed
    to give \textbf{(B)}. The first anti addition is highly steroselective to give
    the E (trans) isomer but it is not stereospecific.

  \item Addition of Hydrogen Chloride.

    \img{3-3}

    The reaction can be controlled to stop after this stage as further addition
    of HCl is slower. The addition is normally anti, resulting in the E (trans)
    isomer.

  \item Addition of HCl in the presence of peroxides

    \img{3-4}

    This reaction occurs by a free radical chain reaction process.

    Initiation:

    \img{3-5}

    Propagation:

    \img{3-6}

    This reaction gives the anto-markovnikov addition as syn addition corrus to give the E isomer since the reaction occurs via the less hindered, trans form of
    vinylic radiocal. \textbf{(A)}.

    \img{3-7}


  \item Hydration of alkynes by electrophilic addition of water
    - useful synthesis of ketones.

    \img{3-8}

    Mechanism:

    Formation of Enol:

    \img{3-9}

    Tautomerisation of the enol to the keto form

    \img{3-10}

    An alterate mechanism can dbe drawn using \ce{HgSO4}/\ce{H2O}:

    \img{3-11}

    Reduction of alkynes to alkenes takes palce via 2 important reactions.

  \item Catalytic hydrogenation using a Lindlar Catalyst - Forms the Z (cis) alkene

    \img{3-12}

    The lindlar catalyst is a poisoned palladium catalyst, it allows syn addition.
    \ce{C=C} is not further reduced. Another variant of the catalyst is:
    Pd/\ce{BaSO4}/quinoline. An industrial use of the catalyst is in making "cis"
    jasmone which is used in perfumery.

  \item Reduction of the \ce{C\equiv C} using disolving metal reduction (Birch
    Reduction)

    Forms the E (trans) alkene.

    \img{3-13}

    Mechanism: Sodium in liquid ammonia is a source of solvated electrons.

    \img{3-14}

  \item The deprotonation of terminal alkynes - Acetylide anions as carbon
    nucleophiles.

    The acidity of a hydrogen atom attached to carbon varies with the degree of
    s character of the C atom. Sp\super{3} (s character 25 \%) < sp\super{2} (
    s character 33.3 \%) < sp (s character 50 \%) and consequently terminal
    alkynes an be deprotonated by a stron gbase such as:

    \begin{itemize}
      \item \ce{Na+NH2-} - sodium amide
      \item \ce{nBuLi} - n-butyl lithium
      \item \ce{R-MgX} - grignard reagent
    \end{itemize}

    Therefore:

    \img{3-15}

    Or for synthesis:

    \img{3-16}

    Acetylide anions are useful for making \ce{C-C} bonds with electrophilic
    reagents.

    \img{3-17}

    Examples of reagents:

    \begin{enumerate}[label=\roman*)]

      \item Alkyl Halides
        \img{3-18}

      \item Epoxide ring opening
        \img{3-19}

    \end{enumerate}

    The deprotonated alkyne can then perform nucleophilic attack on the
    carbonyl groups of aldehydes and ketones.

    On ketones:

    \img{3-20}

    Therefore:

    \img{3-21}

    On aldehydes:

    \img{3-22}

    \ce{Me3Si-C\equiv H} - Trimethylsilyacetene

    This is a useful compound as it allows addition on each C atom of the alkyne
    unit in turn.

    \img{3-23}

\end{enumerate}