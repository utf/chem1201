% !TEX root = ../main.tex
\section{Alkanes}

\subsection{Preamble}

Rings, double bons and other atoms in organic molecules introduce degrees of
unsaturation or double bond equivalents. The formula for working out the degrees
of unsaturation is:

\begin{center}
  $DBE=\dfrac{(2 +(2\times\# Carbons) + \# Nitrogens - \# Hydrogens - \# Halogens)}{2}$
\end{center}

\subsection{Reactions of Alkanes}

The simplest saturated hydrocarbon is methane, \ce{CH4}, which has bon angles of
109\de 28'. Alkenes are made up of tetrahedral sp\super{3} hybridised carbon atoms
covalently bonded with hydrogen. They undergo two elementary reactions:

\subsubsection{Heterolytic Fission}

\img{1-1}

\begin{enumerate}[label=\alph*)]

  \item\
    \img["Superacid" mixture]{1-2}

  \item\

    \img[A trigonal planar, carbocation is formed as the reactive
     intermediate]{1-3}

    Note that charge is conserved and only the tertiary methane hydrogen is
    removed. This is because of the stabilisation effect from R groups.
    Primary carbocations (\ce{R-C+H2}) are higher in energy than secondary
    (\ce{R2C+H}) which are in turn higher than tertiary (\ce{R3C+}) where R
    is an inductively electron releasing alkyl group (e.g. \ce{CH3, CH2CH3}
    etc). This effect can be explained by hyperconjugation, as in a carbocation
    there will be a sp\super{2} hybridised vacant p orbital that can accept
    electrons coming out of the plane of the trigonal planar molecule.
    Electron donation occurs from the \ce{C-H} bond, in the same plane as the
    sp\super{2} orbital, into the empty p orbital.

    \img{1-4}

\end{enumerate}

\subsubsection{Homolytic Fission}

For example free radical chain reactions which are of preparative value.

\img{1-5}

E.g. Free radical halogenation of methane:

\img{1-6-1}

Initiation

\img{1-6}

Propagation

\img{1-7}

Termination

\img{1-8}

Termination reactions require another photon and account for less than 2\% of
the product.

In the free radical halogenation of methane, the propagation steps take place
10\super{6} times for each initiation step. Additionally, other products are formed
when more H's are replaced.

When using isobutane (2-methylpropane) we see interesting selectivity.

\img[Isobutane]{1-9}

With \ce{Br2} the reaction is regiospecific - a single regioisomer is formed.

\img{1-10}

With \ce{Cl2} the reaction is regioselective meaning that there is a preference
for one regiosomer rather than the others.

\img{1-11}

These regio effects occur because the chlorine is much more reactive and hence
less discriminating in hydrogen abstraction than the brome atom
(more electronegative). One \ce{H-Cl} bond formed for Cl is -431 kJmol\super{-1}
and for Br is -366 kJmol\super{-1} and the tertiary C-H bond requires
+397 kJmol\super{-1} to break. Hence with Cl the reaction is exothermic
and with Br it is endothermic meaning the bromine reaction requires energy,
this in turn makes the reaction more discriminating.

Breaking one primary bond needs +423 kJmol\super{-1} and therefore, as reactions
follow the path of least energy and the order of radical stability is 3\de\
> 2\de\ > 1\de , the reaction proceeds above.
