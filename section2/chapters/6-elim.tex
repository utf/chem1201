% !TEX root = ../section2.tex
\section{Elimination Reactions}

Elimination reactions of HX from haloalkanes provide a useful rate for alkene
synthesis.

\begin{figure}[H]
  \centering
  \ce{H3C-CH2-CH2Cl ->[\ce{Na+OEt-}][\ce{\Delta H}] H3C-CH=CH2 + EtOH + NaCl}
\end{figure}

Here, at higher temperatures, sodium ethoxide functions as abase rather than
a nucleophile. If necessary more hindered and less nucleophilic bases can be used,
e.g. \ce{(CH3)3C-O- + K+}. THere are many parallels with S\sub{N}1 and S\sub{N}2
reactions as there are E1 eliminations and E2 eliminations.

\subsection{Bimolecular E2 Eliminations}

For the general case of:

\begin{figure}[H]
  \centering
  \ce{R-CH2-CH2-X ->[\ce{B-}][\ce{-HX}] R-CH=CH2 + BH + X-}
\end{figure}

\textbf{Kinetics}

The rate is proprotional to both the concentration of the haloalkane and base.

\begin{figure}[H]
  \centering
  \ce{Rate = k[RCH_2CH_2X][B^-]}
  \caption*{2\super{nd} order kinetics}
\end{figure}

Since both the base and the alkyl halide are involved in the RDS it is a bimolecular
reaction.\\

\textbf{Stereochemistry}

E2 elimination reactions are often stereospecific and lead to a single geometrical
isomer of an alkene. The conformation in which the base removes the hydrogen
and the leaving group is trans coplanar and is favoured in the transition state.
This is called the trans anti-periplanar position.

As in S\sub{N}2, there is a transition state where bonds are broken and formed in
a concerted fashion.

\img{4-19}

The requirement for a trans anti-periplanar reaction is:

\img[Transition state]{4-20}

\begin{itemize}
  \item the trans relationship of H and X minimises steric replusiion between
    \ce{B-} and \ce{X}.
  \item the coplanar relationship maximises orbital overlap leading to a
    $\pi$ bond with perfect sideways overlap of p orbitals.
\end{itemize}

i.e.

\img{4-21}

\textbf{Energy Profile Diagram}

\img{4-21-2}

\textbf{Regioselectivity}

The preffered product of an alkyl halide elimination reaction is the more
substituted alkene.

\img{4-22}

This is because the more substituted alkene places more of the substituents
around the double bond at 120\de\ and hence constitutes the lower energy
pathway since the transition state resembles the products.
The trans antiperiplanar requirement for E2 elimination allows us to predict
regioselectivity as well as relative rates. For example in the
conformer of:

\img{4-22-1}

there is unfavourable syn interaction between the two phenyl groups:

\img{4-22-2}

This is absent in the opposite conformer and hence the transition state for E2
elimination is higher in energy and the reaction is 50 times slower.

\subsection{Unimolecular E1 Elimination}

\img{4-23}

\textbf{Kinetics}

The rate is proportional to substrate concentration [RX] only and is independent
of [\ce{EtO-}].

\begin{figure}[H]
  \centering
  $Rate = k[RX]$
  \caption*{1\super{st} order kinetics.}
\end{figure}

As with S\sub{N}1 we have a unimolecular reaction where ionisation of the \ce{C-X}
bond is the RDS.\\

\textbf{Mechanism}

\img{4-24}

The base can assist in the fast step of proton removal from the carbocation.\\

\textbf{Energy Profile Diagram}

\img{4-24-2}

\textbf{Characteristics}

E1 eliminations, like S\sub{N}1 are favoured when X is a very good leaving group,
which promotes heterolysis. Additionally when \ce{R+} is a low energy carbocation
and therefore easy to form (e.g. 3\de\ carbocation) or stabilised by
delocalisation (e.g. allyic, \ce{C=C-C+} or benzylic, \ce{Ph-C+R2})

Note: Nucleophilic substition can be inhibited by use of a bulky base which is a
poor nucleophile.