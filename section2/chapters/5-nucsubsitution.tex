% !TEX root = ../section2.tex
\section{Nucleophilic Substitution}

Nucleophilic substitution involves displacement of a good leaving group X as
\ce{X-} by a nucleophile Nu. The nucleophile can be neutral:

\begin{center}
   \ce{Nu + R-X -> [Nu-R]+ + X-}
\end{center}

Or negatively charged:

\begin{center}
   \ce{Nu- + R-X -> Nu-R + X-}
\end{center}

However they always possess a free lone pair of electrons to donate. Examples
include:

\begin{table}[H]
  \centering
  \begin{tabular}{| c | c |}
  \hline
    \textbf{Neutral} & \textbf{Negatively Charged} \\ \hline
    \ce{N(CH3)3} & \ce{Cl-} \\ \hline
    \ce{PR3} & \ce{Br-} \\ \hline
    \ce{SR2} & \ce{I-} \\ \hline
    \ce{OH2} & \ce{N3-} $\equiv$ \ce{N^{-}=N^{+}=N^{-}} \\ \hline
    \ce{H3C-O-H} & \ce{CN-} \\ \hline

  \end{tabular}
\end{table}

\subsection{Trends in Nucleophilicity}

Electronegativity increases across a row of the periodic table \ce{R-NH-R} is
a better nucleophile than \ce{R-O-R}. Additionally descending down a group,
nucleophilicty increases since there are more electrons screening the nucleus,
hence the valence electrons are further away  and more polarisable. If a nucleophile
is used as the solvent for a reaction (e.g. \ce{H2O} or \ce{H3C-OH}, etc), the
nucleophilic substitution is called a solvolysis.

\subsection{Trends in leaving group ability}

\begin{center}
  Tosylate > \ce{R-I} > \ce{R-Br} > \ce{R-O+H2} > \ce{R-Cl}
\end{center}

This is found by experiments measuring rates of reactions.

\subsection{Nucleophilic Substitution Reactions}

The following nucleophilic substitution reactions are particularly important
for organic synthesis.

\begin{enumerate}[label=\alph*)]

  \item Williamson Ether Synthesis

    This is a good general method for the preparation of unsymmetrical ethers.
    It works well with primary alkyl halides (\ce{CH3-X} and \ce{R-CH2-X}) but
    is problematic for hindered tertiary alkyl halides (\ce{R3C-X}). In general:

    \img{4-3}

    Example

    \img{4-4}

  \item Cyanide Anion - Very useful for making \ce{C-C} bonds

    \begin{tabbing}
      E.g.~~~~~ \= \ce{K+ + C\tbond N + H3CCH2-I -> H3C-CH2-C\tbond N} \\
      Then \> \ce{H3CCH2-C\tbond N ->[\ce{H2}][\ce{Pd}] H3C-CH2-CH2-NH2} \\
      Hence \> \ce{R-X -> R-CH2NH2} ~~~ in 2 steps!
    \end{tabbing}

  \item Acetylide Anions - Recap!
    \img[\ce{C=C} bond formation!]{4-5}

\end{enumerate}

\subsection{Bimolecular \texorpdfstring{S\sub{N}2}\ \ Nucleophilic Substitution Reactions}

The S\sub{N}2 reaction is a one step process in which the bond to the nucleophile
is being formed at the same time as the bond the leaving group is being broken.

\img{4-6}

E.g.

\img{4-7}

\textbf{Kinetics}

For the S\sub{N}2 reaction, we find the rate to be proportional to both the
concentrations of the nucleophile [Nu] and the concentration of the alkyl halide
[\ce{R-X}]. Hence

\begin{figure}[H]
  \centering
  $Rate = k[Nu][R-X]$
  \caption*{2\super{nd} order kinetics.}
\end{figure}

Hence two entities are involved in the rate determining step and the reaction is
bimolecular.\\

\textbf{Steroeochemistry}

If we start with a chiral haloalkane, inversion of configuration (Walden Inversion)
is observed since the new bond is being formed at 18\de\ to the bond that is breaking.

\img{4-8}

\textbf{Energy Profile Diagram}

\img{4-8-2}

The experiment to prove that each S\sub{N}2 displacement proceeds with inversion
of configuration was carried out by Huges and Ingold using \ce{^{-}I^{*128}}

\img{4-9}

The reaction gives a racemic product since the iodide anion \ce{I^{*-}} reacts
equally well with both enantiomers. THe optical activity disappears when 50\%
of the starting iodide has undergone substitution. The measured rates are:

\begin{figure}[H]
  \centering
  $Rate\ of\ racemisation = 2 \times rate\ of\ exchange\ of\ Iodide\ by\ I^{*128}$
\end{figure}



As expected from the above analysis, S\sub{N}2 reactions are favoured by
\begin{itemize}
  \item good nucleophiles
  \item good leaving groups
  \item less hindered haloalkane substrates.
\end{itemize}

\subsection{Unimolecular \texorpdfstring{S\sub{N}2}\ \ Nucleophilic Substitution Reactions}

The S\sub{N}1 reaction is a two step process consisting of a slow ionisation
of the substrate, followed by a rapid attack of the nucleophile.

\img{4-10}

E.g.

\img{4-11}

\textbf{Kinetics}

The rate is proportional to the concentration of \ce{R3CX} and independent of
the nucleophile:

\begin{figure}[H]
  \centering
  $Rate = k[R_3CX]$
  \caption*{2\super{nd} order kinetics.}
\end{figure}

Hence only one entity is involved in the rate determining step and the reaction
is unimolecular.

\textbf{Stereochemistry}

If we start with a chiral haloalkane and ionise it by heterolysis, a trigonal
planar sp\super{2} hybridised carbocation intermediate is formed. Since attack of
the nucleophile can occur equally from both sides of this planar intermediate, a
racemic mixture of the product is formed.

\img{4-12}

\textbf{Energy Profile Diagram}

\img{4-12-2}

Carbocationic intermediates, unlike transition states, can undergo some competing
reactions.

E.g. Rearrangement:

\img{4-13}

or proton loss (E\sub{1} elimination):

\img{4-14}

As expected from this analysis, SN\sub{1} reactions are:

\begin{itemize}
  \item not influenced by the nature of the nucleophile.
  \item favoured by good leaving groups to form the carbocation. Sometimes \ce{
    Ag+} is used to favour this pathway: \ce{R3C-Cl + Ag+ -> R3C+ + AgCl}
  \item favoured by substrates that can generate low energy carbocations.

    E.g. Tertiary carbocations via hyperconjugation

    \img{4-15}

    or carbocations stabilised by resonance

    \img{4-16}

\end{itemize}