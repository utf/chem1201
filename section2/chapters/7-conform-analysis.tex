% !TEX root = ../section2.tex
\section{Conformational Analysis}

A conformational change is a change in the 3 dimensional shape of a molecule that
can be achieved without breaking or reforming any bonds.
In acyclic systems, different conformations of a molecule can interconvert rapidly
at room temperature by rotation about one or more single bonds. There are 2 ways
of representing conformers. Sawhorse or Newman projections, respectively:

\img{4-25}

The torsion angle (or dihedral angle), $\Im$, is the angle between opposite ends
of the single bond being examined in a Newman projection. E.g.

\img[Do not confuse with the bond angle]{4-26}

Low energy conformations occur at torsion angles of 60\de\, 180\de\ and 300\de\
(staggered bonds). High energy conformers occur at torsion angles of 0\de\,
120\de\ and 240\de\ (eclipsed bonds).The staggered conformation will always be
lower in energy as pairs of electrons in the \ce{C-H} bonds repel each other
more when eclipsed. Additionally the antibonding orbital of the antiperiplanar
\ce{C-H} bond stabilises the staggered conformer.

\img{4-27}

Replacement of one or more hydrogen atoms in ethane by another group which will
always be larger, leads to steric repulsion or steric strain. This is the extra
internal energy of the molecule when the valence shells of two atoms are competing
for the same space in a molecule. In a molecule with two groups replaced, the lower
energy conformer is the antiperiplanar conformer with a torsion angle of 180\de\
between the two methyl groups. Gauche or synclinal staggered conformers are higher
in energy than the anti conformer. $\Im$ = 60\de\ or 300\de . Eclipsed conformers
have the highest energy.

\subsection{The Cyclohexane Ring}

Cyclohexane:

\img[a = axial substituent and e = equatorial substituent]{4-28}

Boats are higher in energy than chairs:

\img[Bowpoint interaction \ce{e-e} repulsion]{4-29}

The chair is a useful way of representing cyclohexene in its lowest energy conformer:

\img{4-30}

A general rule for a compound containing a conformationally mobile hexane ring is
that the lowest energy conformer will have the larger substituent equatorial. This
is because substituents in an axial position can suffer from 1,3 diaxial
interactions. E.g.

\img{4-31}

As the substituents get larger, the more chance they will be in the equatorial
position.

For distributed rings, the larger group will always preffer to occupy the equatorial
position. For comparison, a methyl group is larger than a bromine atom.

In order to work out if a compound is trans or cis, check to see if the substituents
are both pointing in the same direction with or against the plane. E.g.

\img[Different direction, therefore trans.]{4-32}